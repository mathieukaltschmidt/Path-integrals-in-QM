\section{Der Harmonische Oszillator}
\begin{frame}{Der Harmonische Oszillator}
\metroset{block=fill}
\begin{block}{\hfill Anwendung des erlernten Formalismus \hfill}
Wir wollen nun das Erlernte benutzen um die Energie-Eigenwerte für das  Potential des harmonischen Oszillators zu bestimmen.
\end{block}
Wir kennen den Lagrangian für dieses Problem: 
\begin{align*}
	\mathcal{L} = \frac{m}{2}\dot{q}^2 - \frac{m\omega^2}{2} q^2
\end{align*}

Wir schreiben die Pfade als Abweichungen vom klassischen Pfad:
\begin{align*}
	q(t) = q_{\text{cl}}(t) + \eta(t)
\end{align*}
Die Wirkung des klassischen Beitrags ist bekannt:
\begin{align*}
	\mathcal{S}[q_{\text{cl}}] = \frac{m\omega}{2\sin(\omega T)}\left[(q_T^2+q_0^2)\cos(\omega T)- 2q_Tq_0\right]
\end{align*}
\end{frame}

\begin{frame}{Mehler-Formel und Energiespektrum}
%\metroset{block=fill}
\begin{block}{\hfill Mehler-Formel für den harmonischen Oszillator\hfill }
	\begin{align*}
	\hspace{-0.5cm}
		\braket{q_b,T}{q_a,0} = \sqrt{\frac{m\omega}{2\pi\mathrm{i}\hbar \sin(\omega T)}} \exp\left(\frac{\mathrm{i}m\omega}{2\hbar\sin(\omega T)}\left[(q_T^2+q_0^2)\cos(\omega T)- 2q_Tq_0\right] \right)
	\end{align*}
\end{block}

\vfill
\begin{block}{\hfill Energiespektrum \hfill}
Wir bilden die Spur des Zeitentwicklungsoperators:
\begin{align*}
	\operatorname{Tr}(U(T,0)) = \operatorname{Tr}\left(\exp\left(-\frac{\mathrm{i}}{\hbar}\hat{\text{H}}T\right)\right) 
	= \sum_k \exp\left(-\frac{\mathrm{i}}{\hbar}E_k T \right)
\end{align*}
\end{block}	
\end{frame}

\begin{frame}{Mehler-Formel und Energiespektrum}
\begin{align*}
\operatorname{Tr}(U(T,0)) &= \int_{-\infty}^{\infty} \dd q \braket{q,T}{q,0}	\\
&=  \sqrt{\frac{m\omega}{2\pi\mathrm{i}\hbar\sin(\omega T)}} \int_{-\infty}^{\infty} \dd q  \exp\left(\frac{\mathrm{i}m\omega}{2\hbar\sin(\omega T)}\left[2q^2(\cos(\omega T)- 1)\right] \right) \\
&= \frac{1}{2\mathrm{i}\sin\left(\frac{\omega T}{2}\right)} \\
&= \frac{\exp\left(-\mathrm{i}\frac{\omega T}{2}\right)}{1-\exp(-\mathrm{i}\omega T)}	\\
&= \sum_{k=0}^{\infty} \exp\left(-\frac{\mathrm{i}}{\hbar}\hbar\omega\left(k+\frac{1}{2}\right)T\right)
\end{align*}
Dies liefert uns das gewünschte, bereits bekannte Energie-Spektrum des Harmonischen Oszillators!

\end{frame}


\section{Der klassische Grenzfall}
\begin{frame}{Der klassische Grenzfall}

\begin{itemize}
	\item Klassischer Limes: $\hbar \rightarrow 0$
\end{itemize}
\begin{align*}
	\phi[q] \sim \exp(\frac{\mathrm{i}}{\hbar} \mathcal{S}[q])
\end{align*}
\begin{itemize}
	\item Starke Oszillation im Phasenfaktor
	\item Für den klassischen Pfad ist $\mathcal{S}$ stationär!
\end{itemize} 
\metroset{block=fill}
\begin{block}{\hfill Fazit \hfill}
\begin{itemize}
	\item Der dominante Pfad für den Limes $\hbar\rightarrow 0$ ist gerade der Klassische!
	\item  Die schnelle Oszillation des Exponenten führt zu "destruktiver Interferenz" von den Beiträgen der Pfade, die weit von der klassischen Lösung entfernt sind.	
\end{itemize}
\end{block}

\end{frame}

\begin{frame}{Beiträge zur Übergangsamplitude}
\begin{figure}
\begin{tikzpicture}
\centering

%Koordinatensystem
\draw[->, thick] (0,0) to (10,0);
\node at (10.3,0) {\Large$t$};
\draw[->, thick] (0,0) to (0,6.5);
\node at (0,6.8) {\Large$q$};

%Punkte
\filldraw (2,2)   circle (2pt);
\node at (1.6,1.8) {\Large$a$};
\filldraw (8,5)   circle (2pt);
\node at (8.4,5.2) {\Large$b$};

%Kurven
\draw [-, bend angle = 65, bend left, thick] (2,2) to (8,5);
\draw [-, bend angle=50, bend left, thick, color = myred] (2,2) to (8,5);
\node at (3,5.9) {\color{myred}\Large  q$_{\text{cl}}$(t)};
\draw[->, dashed, color = myred] (3.5,5.7) to (4.5,5);

\draw [-, bend angle = 35, bend left, thick] (2,2) to (8,5);
\node at (5.9,4) {\Large  $\tilde{\text{q}}$(t)};
\draw[->, dashed,] (5.4,4) to (4.5,4.35);

\draw [-, bend angle = 35, bend right, thick] (2,2) to (8,5);
\draw [-, bend angle = 50, bend right, thick] (2,2) to (8,5);	

\end{tikzpicture}
\caption{Abweichungen zum klassischen Pfad}
\end{figure}
\end{frame}
	

\section{Weiterführende Anwendungen}

\begin{frame}{Mathematisches Tool: Wick-Rotation}
Als \textbf{Wick-Rotation} bezeichnen wir die analytische Fortsetzung
\begin{align*}
	t \longrightarrow \operatorname{e}^{-\mathrm{i}\varphi}\tau \qquad \varphi: 0 \rightarrow \frac{\pi}{2}
	\end{align*}

\begin{figure}
\begin{tikzpicture}
\centering
\filldraw[color=gray!20] (2.5,2.5) circle (2cm);
\draw[dashed] (2.5,2.5) circle (2cm);
\draw[->, myred, ultra thick] (4.5,2.5)  arc[start angle=0, end angle=-89,radius=2cm] -- (2.5,0.5);
\draw[->, myred, ultra thick] (0.5,2.5)  arc[start angle=180, end angle=91,radius=2cm] -- (2.5,4.5);

%Koordinatensystem mit Pfeilen
\draw[->, thick] (0,2.5) to (5,2.5);
\node at (5.7,2.5) {$\mathfrak{Re}(t)$};
\draw[->, ultra thick] (0.49,2.5) to (0.5,2.5);
\draw[->, ultra thick] (1.49,2.5) to (1.5,2.5);

\draw[->, ultra thick] (3.49,2.5) to (3.5,2.5);
\draw[->, ultra thick] (4.49,2.5) to (4.5,2.5);

\draw[->, thick] (2.5,0) to (2.5,5);
\node at (2.5,5.4) {$\mathfrak{Im}(t)$};
\draw[->, ultra thick, color =myred] (2.5,3.76) to (2.5,3.75);
\draw[->, ultra thick, color = myred] (2.5,1.26) to (2.5,1.25);

\end{tikzpicture}
\caption{Visualisierung der Wick-Rotation}
\end{figure} 	
\end{frame}
\begin{frame}{Pfadintegrale in der Statistischen Physik}

\begin{itemize}
	\item Betrachte nur Pfade, welche bei der selben Konfiguration starten und auch wieder enden:
	\begin{align*}
		Z = \int \mathcal{D}q \ \exp\left(-\frac{1}{\hbar}\mathcal{S}_{\text{eukl.}}[q]\right)
	\end{align*}
	\item Dies entspricht der kanonischen Zustandssumme mit 
	\begin{align*}
		\frac{1}{T} = \frac{k_B\tau}{\hbar}
	\end{align*}
	\item Insgesamt gilt:
	\begin{align*}
		Z = \int \dd q \ U(q,\beta\hbar;q) = \sum_k\bra{k} \int_q \dd q \ \operatorname{e}^{-\beta E_k}\ket{q}\braket{q}{k} = \sum_k \operatorname{e}^{-\beta E_k}
	\end{align*}
\end{itemize}
\end{frame}

\begin{frame}{Pfadintegrale in der Quantenfeldtheorie}
In der Quantenfeldtheorie wird man häufig mit Pfadintegralen konfrontiert. \\
\begin{itemize}
	\item Man berechnet zum Beispiel \textbf{Erwartungswerte}:
\begin{align*}
	\langle F \rangle = \frac{\int\mathcal{D}\phi \ F[\phi]\ \operatorname{e}^{-\frac{1}{\hbar}\mathcal{S}[\phi]}}{\int\mathcal{D}\phi' \ \operatorname{e}^{-\frac{1}{\hbar}\mathcal{S}[\phi']}}
\end{align*}
\item oder die sog. \textbf{Korrelationsfunktionen}:
\begin{align*}
\left\langle \phi(x_1) \phi(x_2) \ldots \phi(x_n)\right\rangle
=\frac{\int \mathcal D \phi \; \operatorname{e}^{-\frac{1}{\hbar}\mathcal{S}[\phi]}\phi(x_1)\ldots \phi(x_n)}{\int \mathcal D \phi' \; \operatorname{e}^{-\frac{1}{\hbar}\mathcal{S}[\phi']}}
\end{align*}
\end{itemize}
\end{frame}

\begin{frame}
\begin{block}{\hfill \Large Vielen Dank für eure Aufmerksamkeit! \hfill}
	
\end{block}

\end{frame}