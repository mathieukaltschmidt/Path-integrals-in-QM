\section{Kommutatorrelation im Pfadintegralformalismus}
\begin{frame}{Wo versteckt sich die Kommutatorrelation?}
\metroset{block=fill}
\begin{block}{Ein grundlegendes Problem?}
	Ein zentraler Fakt in der Quantenmechanik war bisher:
	\begin{align*}
		[p,q] = -\mathrm{i}\hbar
	\end{align*}
	Wo finden wir die kanonische Kommutatorrelation im Pfadintegral-Formalismus?
\end{block}
\end{frame}

\begin{frame}{Wo versteckt sich die Kommutatorrelation?}
\textbf{Vorgehensweise, basierend auf \cite{Ong}:}
\begin{enumerate}
	\item Aufspalten des Pfadintegrals im Zeitintervall $[0,T]$:\begin{align*}
	\int_{(q_a,0)}^{(q_b,T)} \dd q \operatorname{e}^{\mathrm{i}\mathcal{S}}q(t) &= \int \dd q \braket{q_b,T}{q,t}q\braket{q,t}{q_a,0} \\ &= \int \dd q \bra{q_b,T}\ \tilde{q}(t)\ \ket{q,t}\braket{q,t}{q_a,0} \\ &= \bra{q_b,T}\ \tilde{q}(t)\ \ket{q_a,0}
		\end{align*}
	\item Analog: Produkt von zwei eingefügten q`s:
	\begin{align*}
		\int_{(q_a,0)}^{(q_b,T)} \dd q \operatorname{e}^{\mathrm{i}\mathcal{S}}q(t)q(t') = \bra{q_b,T} \underbrace{\ \hat{\operatorname{T}}\ \left[\hat{q}(t) \ \hat{q}(t')\right] \ }_{= \theta(t-t')q(t)q(t') + \theta(t'-t)q(t)q(t)}\ket{q_a,0}
	\end{align*}
	\item Partielle Integration liefert:
	\begin{align*}
\left\langle\psi_b\left|\frac{\delta F}{\delta q_k}\right|\psi_a\right\rangle = -\frac{\mathrm{i}}{\hbar} \left\langle\psi_b\left|F \frac{\delta \mathcal{S}}{\delta q_k}\right|\psi_a\right\rangle
	\end{align*}
\end{enumerate}
\end{frame}

\begin{frame}{Wo versteckt sich die Kommutatorrelation?}
\textbf{Vorgehensweise, basierend auf \cite{Ong}:}
\begin{enumerate}
	\item[4.] Diskretisiere $\mathcal{S}=\sum_k \mathcal{S}(q_{k+1},q_k) = \frac{m\varepsilon}{2}\left[\frac{q_{k+1}-q_k}{\varepsilon}\right]^2 - \varepsilon V(q_{k+1})$:
	\begin{align*}
		\mathrm{i}\hbar\frac{\delta F}{\delta q_k} \overset{S}{\longleftrightarrow} F\left[\frac{\delta\mathcal{S}(q_{k+1},q_k)}{\delta q_k} + \frac{\delta\mathcal{S}(q_{k},q_{k-1})}{\delta q_k}\right] 
	\end{align*}
	\item[5.] Wähle $F=q_k$ und vernachlässige Terme $\mathcal{O}(\varepsilon)$:
	\begin{align*}
		\underbrace{m\left(\frac{q_{k+1}-q_k}{\varepsilon}\right)}_{p_k}q_k - \underbrace{m\left(\frac{q_{k}-q_{k-1}}{\varepsilon}\right)}_{p_k}q_k  \overset{S}{\longleftrightarrow} -i\hbar
	\end{align*}
\end{enumerate}
\metroset{block=fill}
\begin{block}{\hfill Fazit \hfill}
	 Die Kommutatorrelation versteckt sich also in der Diskretisierung und der sukzessiven Anwendung des Zeitordnungsoperators $\hat{\operatorname{T}}$! 
\end{block}
\end{frame}