\section{Von der klassischen Mechanik zur Quantenmechanik}
\begin{frame}{Klassische Dynamik von Teilchen}
\begin{figure}
\begin{tikzpicture}
\centering

%Koordinatensystem
\draw[->, thick] (0,0) to (10,0);
\node at (10.3,0) {\Large$t$};
\draw[->, thick] (0,0) to (0,6.5);
\node at (0,6.8) {\Large$q$};

%Punkte
\filldraw (2,2)   circle (2pt);
\node at (1.6,1.8) {\Large$a$};
\filldraw (8,5)   circle (2pt);
\node at (8.4,5.2) {\Large$b$};

%Trajektorie
\draw[-, thick] (2,2) to [curve through = {(3, 1.9) (3.7,3.4)  (4.5,5)}] (8,5);

\end{tikzpicture}
\caption{Trajektorie eines klassischen Punktteilchens}
\end{figure}
\end{frame}

\begin{frame}{Klassische Dynamik von Teilchen}
\begin{itemize}
	\item Trajektorien ergeben sich aus dem \textbf{Wirkungsprinzip}:
	\begin{align*}
	\text{Wirkung } S = \int_{t_1}^{t_2} \dd t \ \mathcal{L}(q(t),\dot{q}(t)) \text{ wird  minimiert.} 
	\end{align*}
	\item Erste Variation $\delta S$ soll verschwinden:
	\begin{align*}
		\delta S = \int_{t_1}^{t_2} \dd t \frac{\partial \mathcal{L}}{\partial q} \delta q + \frac{\partial \mathcal{L}}{\partial \dot{q}} \delta \dot{q} \overset{!}{ = } 0
	\end{align*}
	\item Ausführen der Variation liefert \textbf{Euler-Lagrange-Gleichungen:} 
	\begin{align*}
		\frac{\dd}{\dd t}\frac{\partial \mathcal{L}}{\partial \dot{q}} - \frac{\partial \mathcal{L}}{\partial q} = 0
	\end{align*}
\end{itemize}
\end{frame}

\begin{frame}{Bisheriger Zugang zur Quantenmechanik}
\begin{itemize}
	\item Determinismus $\longrightarrow $ Wahrscheinlichkeitsaussagen
	\item Zentrale Objekte sind \textbf{Zustände} $\ket{\psi} \in \mathcal{H}$
	\item Messung verschiedener Observablen mittels selbsadj. \textbf{Operatoren}
\end{itemize}
\begin{center}
	\textbf{Interpretationen der Quantenmechanik}
\end{center}
\begin{enumerate}
	\item Matrizenmechanik  nach Heisenberg et al.
	\begin{align*}
	\frac{\mathrm{d}\hat{\text{A}}_{{\rm H}}}{\mathrm{d}t}=\frac{\mathrm{i}}{\hbar}\left[\hat{\text{H}}_{\rm H},\hat{\text{A}}_{\rm H}\right]+\left( \partial_t \hat{\text{A}}_{\rm S}\right)_{\rm H}
	\end{align*}
	\item Wellenmechanik nach Schrödinger
	\begin{align*}
		\mathrm{i}\hbar \frac{\partial}{\partial t} \ket{\psi(t)} = \hat{\text{H}} \ket{\psi(t)}
	\end{align*}
	\item \textbf{Mein Vortrag:} Pfadintegral-Formalismus nach Feynman et al. 
\end{enumerate}
\end{frame}
\begin{frame}{Entwicklung des Pfadintegral-Formalismus}

\begin{itemize}
	\item Grundlegende Beiträge von \textbf{Gregor Wentzel, Paul Dirac} und insbesondere \textbf{Richard P. Feynman}
\end{itemize}

\begin{figure}[H]
	\begin{minipage}{0.275\textwidth}
	\includegraphics[width = \textwidth]{figures/wentzel}	
	\end{minipage}
	\begin{minipage}{0.315\textwidth}
	\includegraphics[width = \textwidth]{figures/dirac}	
	\end{minipage}
	\begin{minipage}{0.335\textwidth}
	\includegraphics[width = \textwidth]{figures/feynman}	
	\end{minipage}
	
	\caption{Von links nach rechts: G. Wentzel, P. Dirac und R. Feynman}
\end{figure}
\end{frame}

\begin{frame}{Entwicklung des Formalismus}
	\begin{itemize}
		\item \textbf{1924: G. Wentzel} entdeckt 1924 das später nach Feynman benannte Pfadintegral. 
		\vfill 
		\item \textbf{1933: P. Dirac} erkennt die besondere Bedeutung des Wirkungsfunktionals für die klassische Mechanik, vgl. \cite{Dirac1934}.
		 \vfill 
		 \item \textbf{1948: R. Feynman} formuliert seine grundlegende, alternative Formulierung der Quantenmechanik, den Pfadintegral-Formalismus, vgl. \cite{Feynman1948}.
	\end{itemize}
\end{frame}