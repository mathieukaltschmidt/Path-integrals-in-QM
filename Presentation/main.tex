\documentclass[10pt]{beamer}
\usetheme[progressbar=frametitle]{metropolis} 
\setsansfont{Optima}
\setmonofont{Menlo}

%General Information
\title{\LARGE Die Pfadintegral-Formulierung \\ der Quantenmechanik}
\subtitle{\small Vortrag im Rahmen des Quantenmechanik-Seminars bei Prof. Wolschin\\}
\date{21. Dezember 2018}
\author{\normalsize\textbf{Mathieu Kaltschmidt}}
\institute{\normalsize Institut für Theoretische Physik, Universität Heidelberg}

%Useful settings 
%math and theorems
\usepackage{amsmath}
\usepackage{amssymb}

%language settings
\usepackage{fontspec} 
\usepackage{polyglossia}
\setmainlanguage{german}
\setotherlanguages{english}


%useful packages
\usepackage{appendixnumberbeamer}
\usepackage{booktabs}
\usepackage{siunitx}
\usepackage{xcolor}
\usepackage{graphicx}
\usepackage{float}
\usepackage{blindtext}
\usepackage{physics}
\usepackage[labelfont=bf]{caption}



%Some additional color settings
\definecolor{myred}{RGB}{196,19,47} 
\definecolor{myblue}{RGB}{65,105,225}
\definecolor{mydarkblue}{RGB}{1,1,141}
\definecolor{mygray}{RGB}{102, 102, 142}

% nice fracs
\usepackage{xfrac}


%Tikz
\usepackage{tikz}
\usetikzlibrary{shapes, patterns, hobby}
\usepackage{pgfplots}
\usepgfplotslibrary{fillbetween}
\usetikzlibrary{positioning,arrows}






%Main document
\begin{document}

\maketitle

\begin{frame}{Inhaltsverzeichnis}
  \setbeamertemplate{section in toc}[sections numbered]
  \tableofcontents[hideallsubsections]
\end{frame}

%Content
\include{content/einführung}
\include{content/double_slit}
\section{Die Idee des Pfadintegral-Formalismus}
\begin{frame}{Die Idee des Pfadintegral-Formalismus}

\metroset{block=fill}
\begin{block}{Grundlegende Fragestellung}
	Was ist die \textbf{Wahrscheinlichkeitsamplitude} für den Übergang von $(q_a, t_a)$ nach $(q_b, t_b)$ für $t_b>t_a$.  
\end{block}
\textbf{Feynman's Ideen von 1948}
\begin{itemize}
	\item Die Amplitude bezeichnet man als \textbf{Kernel }$K(b,a)$:
	\begin{align*}
		K(b,a) = \sum_{\text{alle Pfade}}\phi[q(t)]
	\end{align*}
	\item Beitrag eines einzelnen Pfades:
	\begin{align*}
		\phi[q(t)] \sim N \cdot \exp\left(\frac{\mathrm{i}}{\hbar}\mathcal{S}[q(t)]\right) 
	\end{align*}
	\item  Wahrscheinlichkeit für den Übergang:
	\begin{align*}
	 P(b,a) = \abs{K(b,a)}^2 
	 \end{align*}
\end{itemize}
\end{frame}

\begin{frame}{Ausgangslage für unsere Herleitung}
Interessant ist die Analyse der \textbf{Übergangsamplitude}:
\begin{align*}
\braket{q_b,t_b}{q_a,t_a} = \bra{q_b}\exp\left(-\mathrm{i} \hat{\text{H}}(t_b-t_a)\right)\ket{q_a}
\end{align*}

\textbf{Einige hilfreiche Identitäten}:
\vfill 
\begin{itemize}
\item Identität im Orstraum: $\mathrm{1\!l} = \int\dd q_k \ \ket{q_k}\bra{q_k} $
\vfill 
\item Baker-Hausdorff-Campbell: $\operatorname{e}^{A+B} = \operatorname{e}^{A}\operatorname{e}^{B}\operatorname{e}^{-\frac{1}{2}[A+B] + \cdots} $
\vfill 
\item Basistransformation: $\braket{p_k}{q_k} = \frac{1}{\sqrt{2\pi}}\operatorname{e}^{-{\mathrm{i}p_kq_k}}$
\end{itemize}
\end{frame}

%Hier startet der Tafelvortrag

%Visualisierung während der Herleitung an der Tafel 
\begin{frame}{Die Idee des Pfadintegral-Formalismus}
\begin{figure}
\centering
\begin{tikzpicture}

%Koordinatensystem
\draw[->, thick] (0,0) to (10,0);
\node at (10.3,0) {\Large$q$};
\draw[->, thick] (0,0) to (0,6.5);
\node at (0,6.8) {\Large$t$};

%Horizontale Linien
\draw[-, thick, dashed] (0,1) to (10,1);
\draw[-, thick, dashed] (0,2) to (10,2);
\draw[-, thick, dashed] (0,3) to (10,3);
\draw[-, thick, dashed] (0,4) to (10,4);
\draw[-, thick, dashed] (0,5) to (10,5);
\draw[-, thick, dashed] (0,6) to (10,6);


%Punkte
\node at (2.95,0.7) {\Large$a$};
\filldraw (3,1)   circle (2pt);
\filldraw (6,2)   circle (2pt);
\filldraw (6.5,3)   circle (2pt);
\filldraw (5,4)   circle (2pt);
\filldraw (4,5)   circle (2pt);
\filldraw (5.5,6)   circle (2pt);
\node at (5.55,6.4) {\Large$b$};

%Verbindungen
\draw[-, thick] (3,1) to (6,2);
\draw[-, thick] (6,2) to (6.5,3);
\draw[-, thick] (6.5,3) to (5,4);
\draw[-, thick] (5,4) to (4,5);
\draw[-, thick] (4,5) to (5.5,6);

%Markierung 
\draw[<->, ultra thick, color = myred] (2.5,3) to (2.5,4);
\node at (2.8,3.5) {\color{myred}\Large$\delta$t};

\end{tikzpicture}
\caption{Konstruktion der "Summe über alle Pfade"}
\end{figure}
\end{frame}

\begin{frame}{Übergang zu zwei und mehr Events}
\begin{figure}
\centering
\begin{tikzpicture}

%Koordinatensystem
\draw[->, thick] (0,0) to (10,0);
\node at (10.3,0) {\Large$q$};
\draw[->, thick] (0,0) to (0,6.5);
\node at (0,6.8) {\Large$t$};


%Domain 
\filldraw[color=myred!30] (5,3) ellipse (3.5cm and 2.5 cm);
\draw[-, thick] (5,3) ellipse (3.5cm and 2.5 cm);

%Horizontale Linie
\draw[-, thick, dashed] (0,3) to (10,3);
\node at (0.3,3.3) {\Large$t_k$};

%Beschriftung 
\node at (8,6) {\Large Ereignisbereich $R$};
\node at (5,4.5) {\Large $R \ ''$};
\node at (5,1.5) {\Large  $R \ '$};

\end{tikzpicture}
\caption{Zerlegung des Ereignisraumes}
\end{figure}
\end{frame}
\begin{frame}{Übergang zu zwei und mehr Events}
\begin{figure}
\centering
\begin{tikzpicture}
%Koordinatensystem
\draw[->, thick] (0,0) to (10,0);
\node at (10.3,0) {\Large$q$};
\draw[->, thick] (0,0) to (0,6.5);
\node at (0,6.8) {\Large$t$};

%Einzelspalt
%\node at (-0.3,3) {$t_c$};
\draw[-, ultra thick] (0,3) to (4.7,3);
\draw[-, ultra thick] (5.3,3) to (10,3);

%Punkte und Verbindungen
\draw[-,thick, color = myred] (2,1) to (5,3);
\draw[-,thick, dashed, bend angle = 20, bend right] (2,1) to (5,3);
\draw[-,thick, dashed, bend angle = 20, bend left] (2,1) to (5,3);
\draw[-,thick, dashed, bend angle = 50, bend right] (2,1) to (5,3);
\draw[-,thick, dashed, bend angle = 80, bend right] (2,1) to (5,3);


\draw[-,thick, color = myred] (5,3) to (4,6);
\draw[-,thick, dashed, bend angle = 20, bend right] (5,3) to (4,6);
\draw[-,thick, dashed, bend angle = 60, bend left] (5,3) to (4,6);
\draw[-,thick, dashed, bend angle = 45, bend right] (5,3) to (4,6);
\draw[-,thick, dashed, bend angle = 20, bend left] (5,3) to (4,6);

\filldraw (2,1) circle (2pt);
\filldraw (5,3) circle (2pt);
\filldraw (4,6) circle (2pt);
\end{tikzpicture}
\caption{Einzelspalt zur Veranschaulichung der Idee von Multi-Events}
\end{figure}
\end{frame}
\begin{frame}{Wichtiges Zwischenergebnis}
\metroset{block=fill}
\begin{block}{Formale Definition des Pfadintegrals}
\begin{align*}
	K(b,a) &= \braket{q_b,t_b}{q_a,t_a}\\
	&= \int_a^b \mathcal{D}[q(t)] \exp\left(\frac{\mathrm{i}}{\hbar}\int_{t_a}^{t_b}\dd t \ \mathcal{L}(q(t),\dot{q}(t))\right)
\end{align*}
\end{block}
Im nächsten Schritt untersuchen wir den Fall eines allgemeinen Hamiltonians der Form:
\begin{align*}
	\hat{\text{H}}(p,q) = \frac{p^2}{2m} + V(q) 
\end{align*}
Wir benötigen die Definition des \textbf{Gaußintegrals}: 
\begin{align*}
	\int_{-\infty}^{\infty}\dd q \exp\left(-\frac{1}{2}a q^2 + b q\right) = \sqrt{\frac{2\pi}{a}}\exp\left(\frac{b^2}{2a}\right) \quad \text{mit } \mathfrak{Re}(a)>0 
\end{align*}
\end{frame}

%Nächster Teil an der Tafel (kurz!!!)

\begin{frame}{Ergebnis der Herleitung}
Endergebnis der Herleitung für einen Hamiltonian der Form $\frac{p^2}{2m} + V(q)$:
\metroset{block=fill}
\begin{block}{Feynman-Kac-Formel}
\begin{align*}
	 \braket{q_b,t_b}{q_a,t_a} &= \lim_{N \rightarrow\infty} \gamma^{N+1}\prod_{k=1}^{N} \int\dd q_k \exp\left(\mathrm{i}\int_{t_a}^{t_b}\dd t \ \mathcal{L}(p,q)\right) \\
	 &\equiv \int_{q(t_a)}^{q(t_b)} \mathcal{D}q(t) \exp\left(\mathrm{i}\int_{t_a}^{t_b}\dd t \ \mathcal{L}(p,q)\right)
\end{align*}	
\end{block}
	
\end{frame} %TODO at blackboard!! TeX it :-)
\section{Kommutatorrelation im Pfadintegralformalismus}
\begin{frame}{Wo versteckt sich die Kommutatorrelation?}
\metroset{block=fill}
\begin{block}{Ein grundlegendes Problem?}
	Ein zentraler Fakt in der Quantenmechanik war bisher:
	\begin{align*}
		[p,q] = -\mathrm{i}\hbar
	\end{align*}
	Wo finden wir die kanonische Kommutatorrelation im Pfadintegral-Formalismus?
\end{block}
\end{frame}

\begin{frame}{Wo versteckt sich die Kommutatorrelation?}
\textbf{Vorgehensweise, basierend auf \cite{Ong}:}
\begin{enumerate}
	\item Aufspalten des Pfadintegrals im Zeitintervall $[0,T]$:\begin{align*}
	\int_{(q_a,0)}^{(q_b,T)} \dd q \operatorname{e}^{\mathrm{i}\mathcal{S}}q(t) &= \int \dd q \braket{q_b,T}{q,t}q\braket{q,t}{q_a,0} \\ &= \int \dd q \bra{q_b,T}\ \tilde{q}(t)\ \ket{q,t}\braket{q,t}{q_a,0} \\ &= \bra{q_b,T}\ \tilde{q}(t)\ \ket{q_a,0}
		\end{align*}
	\item Analog: Produkt von zwei eingefügten q`s:
	\begin{align*}
		\int_{(q_a,0)}^{(q_b,T)} \dd q \operatorname{e}^{\mathrm{i}\mathcal{S}}q(t)q(t') = \bra{q_b,T} \underbrace{\ \hat{\operatorname{T}}\ \left[\hat{q}(t) \ \hat{q}(t')\right] \ }_{= \theta(t-t')q(t)q(t') + \theta(t'-t)q(t)q(t)}\ket{q_a,0}
	\end{align*}
	\item Partielle Integration liefert:
	\begin{align*}
\left\langle\psi_b\left|\frac{\delta F}{\delta q_k}\right|\psi_a\right\rangle = -\frac{\mathrm{i}}{\hbar} \left\langle\psi_b\left|F \frac{\delta \mathcal{S}}{\delta q_k}\right|\psi_a\right\rangle
	\end{align*}
\end{enumerate}
\end{frame}

\begin{frame}{Wo versteckt sich die Kommutatorrelation?}
\textbf{Vorgehensweise, basierend auf \cite{Ong}:}
\begin{enumerate}
	\item[4.] Diskretisiere $\mathcal{S}=\sum_k \mathcal{S}(q_{k+1},q_k) = \frac{m\varepsilon}{2}\left[\frac{q_{k+1}-q_k}{\varepsilon}\right]^2 - \varepsilon V(q_{k+1})$:
	\begin{align*}
		\mathrm{i}\hbar\frac{\delta F}{\delta q_k} \overset{S}{\longleftrightarrow} F\left[\frac{\delta\mathcal{S}(q_{k+1},q_k)}{\delta q_k} + \frac{\delta\mathcal{S}(q_{k},q_{k-1})}{\delta q_k}\right] 
	\end{align*}
	\item[5.] Wähle $F=q_k$ und vernachlässige Terme $\mathcal{O}(\varepsilon)$:
	\begin{align*}
		\underbrace{m\left(\frac{q_{k+1}-q_k}{\varepsilon}\right)}_{p_k}q_k - \underbrace{m\left(\frac{q_{k}-q_{k-1}}{\varepsilon}\right)}_{p_k}q_k  \overset{S}{\longleftrightarrow} -i\hbar
	\end{align*}
\end{enumerate}
\metroset{block=fill}
\begin{block}{\hfill Fazit \hfill}
	 Die Kommutatorrelation versteckt sich also in der Diskretisierung und der sukzessiven Anwendung des Zeitordnungsoperators $\hat{\operatorname{T}}$! 
\end{block}
\end{frame}
\section{Der Harmonische Oszillator}
\begin{frame}{Der Harmonische Oszillator}
\metroset{block=fill}
\begin{block}{\hfill Anwendung des erlernten Formalismus \hfill}
Wir wollen nun das Erlernte benutzen um die Energie-Eigenwerte für das  Potential des harmonischen Oszillators zu bestimmen.
\end{block}
Wir kennen den Lagrangian für dieses Problem: 
\begin{align*}
	\mathcal{L} = \frac{m}{2}\dot{q}^2 - \frac{m\omega^2}{2} q^2
\end{align*}

Wir schreiben die Pfade als Abweichungen vom klassischen Pfad:
\begin{align*}
	q(t) = q_{\text{cl}}(t) + \eta(t)
\end{align*}
Die Wirkung des klassischen Beitrags ist bekannt:
\begin{align*}
	\mathcal{S}[q_{\text{cl}}] = \frac{m\omega}{2\sin(\omega T)}\left[(q_T^2+q_0^2)\cos(\omega T)- 2q_Tq_0\right]
\end{align*}
\end{frame}

\begin{frame}{Mehler-Formel und Energiespektrum}
%\metroset{block=fill}
\begin{block}{\hfill Mehler-Formel für den harmonischen Oszillator\hfill }
	\begin{align*}
	\hspace{-0.5cm}
		\braket{q_b,T}{q_a,0} = \sqrt{\frac{m\omega}{2\pi\mathrm{i}\hbar \sin(\omega T)}} \exp\left(\frac{\mathrm{i}m\omega}{2\hbar\sin(\omega T)}\left[(q_T^2+q_0^2)\cos(\omega T)- 2q_Tq_0\right] \right)
	\end{align*}
\end{block}

\vfill
\begin{block}{\hfill Energiespektrum \hfill}
Wir bilden die Spur des Zeitentwicklungsoperators:
\begin{align*}
	\operatorname{Tr}(U(T,0)) = \operatorname{Tr}\left(\exp\left(-\frac{\mathrm{i}}{\hbar}\hat{\text{H}}T\right)\right) 
	= \sum_k \exp\left(-\frac{\mathrm{i}}{\hbar}E_k T \right)
\end{align*}
\end{block}	
\end{frame}

\begin{frame}{Mehler-Formel und Energiespektrum}
\begin{align*}
\operatorname{Tr}(U(T,0)) &= \int_{-\infty}^{\infty} \dd q \braket{q,T}{q,0}	\\
&=  \sqrt{\frac{m\omega}{2\pi\mathrm{i}\hbar\sin(\omega T)}} \int_{-\infty}^{\infty} \dd q  \exp\left(\frac{\mathrm{i}m\omega}{2\hbar\sin(\omega T)}\left[2q^2(\cos(\omega T)- 1)\right] \right) \\
&= \frac{1}{2\mathrm{i}\sin\left(\frac{\omega T}{2}\right)} \\
&= \frac{\exp\left(-\mathrm{i}\frac{\omega T}{2}\right)}{1-\exp(-\mathrm{i}\omega T)}	\\
&= \sum_{k=0}^{\infty} \exp\left(-\frac{\mathrm{i}}{\hbar}\hbar\omega\left(k+\frac{1}{2}\right)T\right)
\end{align*}
Dies liefert uns das gewünschte, bereits bekannte Energie-Spektrum des Harmonischen Oszillators!

\end{frame}


\section{Der klassische Grenzfall}
\begin{frame}{Der klassische Grenzfall}

\begin{itemize}
	\item Klassischer Limes: $\hbar \rightarrow 0$
\end{itemize}
\begin{align*}
	\phi[q] \sim \exp(\frac{\mathrm{i}}{\hbar} \mathcal{S}[q])
\end{align*}
\begin{itemize}
	\item Starke Oszillation im Phasenfaktor
	\item Für den klassischen Pfad ist $\mathcal{S}$ stationär!
\end{itemize} 
\metroset{block=fill}
\begin{block}{\hfill Fazit \hfill}
\begin{itemize}
	\item Der dominante Pfad für den Limes $\hbar\rightarrow 0$ ist gerade der Klassische!
	\item  Die schnelle Oszillation des Exponenten führt zu "destruktiver Interferenz" von den Beiträgen der Pfade, die weit von der klassischen Lösung entfernt sind.	
\end{itemize}
\end{block}

\end{frame}

\begin{frame}{Beiträge zur Übergangsamplitude}
\begin{figure}
\begin{tikzpicture}
\centering

%Koordinatensystem
\draw[->, thick] (0,0) to (10,0);
\node at (10.3,0) {\Large$t$};
\draw[->, thick] (0,0) to (0,6.5);
\node at (0,6.8) {\Large$q$};

%Punkte
\filldraw (2,2)   circle (2pt);
\node at (1.6,1.8) {\Large$a$};
\filldraw (8,5)   circle (2pt);
\node at (8.4,5.2) {\Large$b$};

%Kurven
\draw [-, bend angle = 65, bend left, thick] (2,2) to (8,5);
\draw [-, bend angle=50, bend left, thick, color = myred] (2,2) to (8,5);
\node at (3,5.9) {\color{myred}\Large  q$_{\text{cl}}$(t)};
\draw[->, dashed, color = myred] (3.5,5.7) to (4.5,5);

\draw [-, bend angle = 35, bend left, thick] (2,2) to (8,5);
\node at (5.9,4) {\Large  $\tilde{\text{q}}$(t)};
\draw[->, dashed,] (5.4,4) to (4.5,4.35);

\draw [-, bend angle = 35, bend right, thick] (2,2) to (8,5);
\draw [-, bend angle = 50, bend right, thick] (2,2) to (8,5);	

\end{tikzpicture}
\caption{Abweichungen zum klassischen Pfad}
\end{figure}
\end{frame}
	

\section{Weiterführende Anwendungen}

\begin{frame}{Mathematisches Tool: Wick-Rotation}
Als \textbf{Wick-Rotation} bezeichnen wir die analytische Fortsetzung
\begin{align*}
	t \longrightarrow \operatorname{e}^{-\mathrm{i}\varphi}\tau \qquad \varphi: 0 \rightarrow \frac{\pi}{2}
	\end{align*}

\begin{figure}
\begin{tikzpicture}
\centering
\filldraw[color=gray!20] (2.5,2.5) circle (2cm);
\draw[dashed] (2.5,2.5) circle (2cm);
\draw[->, myred, ultra thick] (4.5,2.5)  arc[start angle=0, end angle=-89,radius=2cm] -- (2.5,0.5);
\draw[->, myred, ultra thick] (0.5,2.5)  arc[start angle=180, end angle=91,radius=2cm] -- (2.5,4.5);

%Koordinatensystem mit Pfeilen
\draw[->, thick] (0,2.5) to (5,2.5);
\node at (5.7,2.5) {$\mathfrak{Re}(t)$};
\draw[->, ultra thick] (0.49,2.5) to (0.5,2.5);
\draw[->, ultra thick] (1.49,2.5) to (1.5,2.5);

\draw[->, ultra thick] (3.49,2.5) to (3.5,2.5);
\draw[->, ultra thick] (4.49,2.5) to (4.5,2.5);

\draw[->, thick] (2.5,0) to (2.5,5);
\node at (2.5,5.4) {$\mathfrak{Im}(t)$};
\draw[->, ultra thick, color =myred] (2.5,3.76) to (2.5,3.75);
\draw[->, ultra thick, color = myred] (2.5,1.26) to (2.5,1.25);

\end{tikzpicture}
\caption{Visualisierung der Wick-Rotation}
\end{figure} 	
\end{frame}
\begin{frame}{Pfadintegrale in der Statistischen Physik}

\begin{itemize}
	\item Betrachte nur Pfade, welche bei der selben Konfiguration starten und auch wieder enden:
	\begin{align*}
		Z = \int \mathcal{D}q \ \exp\left(-\frac{1}{\hbar}\mathcal{S}_{\text{eukl.}}[q]\right)
	\end{align*}
	\item Dies entspricht der kanonischen Zustandssumme mit 
	\begin{align*}
		\frac{1}{T} = \frac{k_B\tau}{\hbar}
	\end{align*}
	\item Insgesamt gilt:
	\begin{align*}
		Z = \int \dd q \ U(q,\beta\hbar;q) = \sum_k\bra{k} \int_q \dd q \ \operatorname{e}^{-\beta E_k}\ket{q}\braket{q}{k} = \sum_k \operatorname{e}^{-\beta E_k}
	\end{align*}
\end{itemize}
\end{frame}

\begin{frame}{Pfadintegrale in der Quantenfeldtheorie}
In der Quantenfeldtheorie wird man häufig mit Pfadintegralen konfrontiert. \\
\begin{itemize}
	\item Man berechnet zum Beispiel \textbf{Erwartungswerte}:
\begin{align*}
	\langle F \rangle = \frac{\int\mathcal{D}\phi \ F[\phi]\ \operatorname{e}^{-\frac{1}{\hbar}\mathcal{S}[\phi]}}{\int\mathcal{D}\phi' \ \operatorname{e}^{-\frac{1}{\hbar}\mathcal{S}[\phi']}}
\end{align*}
\item oder die sog. \textbf{Korrelationsfunktionen}:
\begin{align*}
\left\langle \phi(x_1) \phi(x_2) \ldots \phi(x_n)\right\rangle
=\frac{\int \mathcal D \phi \; e^{-\frac{1}{\hbar}\mathcal{S}[\phi]}\phi(x_1)\ldots \phi(x_n)}{\int \mathcal D \phi' \; e^{-\frac{1}{\hbar}\mathcal{S}[\phi']}}
\end{align*}
\end{itemize}
\end{frame}

\begin{frame}
\begin{block}{\hfill \Large Vielen Dank für eure Aufmerksamkeit! \hfill}
	
\end{block}

\end{frame}


%Appendix
\appendix
\begin{frame}[shrink=20]{Literaturverzeichnis}
  \nocite{*}
  \bibliography{bibliography/pathintegral.bib}
  \bibliographystyle{plain}
\end{frame}


\end{document}