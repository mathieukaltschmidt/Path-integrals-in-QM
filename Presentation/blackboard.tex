\documentclass{scrartcl}

%include the settings from the presentation 
%math and theorems
\usepackage{amsmath}
\usepackage{amssymb}

%language settings
\usepackage{fontspec} 
\usepackage{polyglossia}
\setmainlanguage{german}
\setotherlanguages{english}


%useful packages
\usepackage{appendixnumberbeamer}
\usepackage{booktabs}
\usepackage{siunitx}
\usepackage{xcolor}
\usepackage{graphicx}
\usepackage{float}
\usepackage{blindtext}
\usepackage{physics}
\usepackage[labelfont=bf]{caption}



%Some additional color settings
\definecolor{myred}{RGB}{196,19,47} 
\definecolor{myblue}{RGB}{65,105,225}
\definecolor{mydarkblue}{RGB}{1,1,141}
\definecolor{mygray}{RGB}{102, 102, 142}

% nice fracs
\usepackage{xfrac}


%Tikz
\usepackage{tikz}
\usetikzlibrary{shapes, patterns, hobby}
\usepackage{pgfplots}
\usepgfplotslibrary{fillbetween}
\usetikzlibrary{positioning,arrows}



\setmainfont{Palatino}

\begin{document}
Interessant ist die Analyse der \textbf{Übergangsamplitude}:
\begin{align*}
\braket{q_b,t_b}{q_a,t_a} = \bra{q_b}\exp\left(-\mathrm{i} \hat{\text{H}}(t_b-t_a)\right)\ket{q_a}
\end{align*}

\textbf{Einige hilfreiche Identitäten}:

\begin{itemize}
\item Identitität im Orstraum: $\mathrm{1\!l} = \int\dd q_k \ \ket{q_k}\bra{q_k} $
\item Baker-Hausdorff-Campbell: $\operatorname{e}^{A+B} = \operatorname{e}^{A}\operatorname{e}^{B}\operatorname{e}^{-\frac{1}{2}[A+B] + \cdots} $
\item Basistransformation: $\braket{p_k}{q_k} = \frac{1}{\sqrt{2\pi}}\operatorname{e}^{-{\mathrm{i}p_kq_k}}$
\end{itemize}
\end{document}