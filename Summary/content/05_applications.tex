\section{Weiterführende Anwendungen}
Wie bereits in der Rechnung zum Harmonischen Oszillator deutlich wurde, liefert der Pfadintegral-Formalismus zwar die gleichen Resultate wie die anderen Zugänge, die Rechnungen sind jedoch sehr aufwendig. Das Wasserstoff-Atom beispielsweise wurde erst knapp 30 Jahre nach Feynmans Originalarbeit im Pfadintegral-Formalismus gelöst. \\
Große Anwendung finden Pfadintegrale jedoch in der Statistischen Physik und in der Quantenfeldtheorie, was nachfolgend anhand einiger Beispiele vorgestellt werden soll.
\subsection{Die Wick-Rotation}
Ein sehr nützliches mathematisches Hilfswerkzeug  ist die sog. Wick-Rotation, welche  es uns ermöglicht, eine analytische Fortsetzung in die komplexe Zahlenebene durchzuführen. Hierzu wird eine Substitution der Form
\begin{align}
	t \longrightarrow \operatorname{e}^{-i\varphi}\tau \qquad\qquad \varphi: 0 \rightarrow \frac{\pi}{2}
\end{align}
durchgeführt. 

\begin{figure}[H]
\centering
\begin{tikzpicture}
\filldraw[color=gray!20] (2.5,2.5) circle (2cm);
\draw[dashed] (2.5,2.5) circle (2cm);
\draw[->, myred, ultra thick] (4.5,2.5)  arc[start angle=0, end angle=-89,radius=2cm] -- (2.5,0.5);
\draw[->, myred, ultra thick] (0.5,2.5)  arc[start angle=180, end angle=91,radius=2cm] -- (2.5,4.5);

%Koordinatensystem mit Pfeilen
\draw[->, thick] (0,2.5) to (5,2.5);
\node at (5.7,2.5) {$\mathfrak{Re}(t)$};
\draw[->, ultra thick] (0.49,2.5) to (0.5,2.5);
\draw[->, ultra thick] (1.49,2.5) to (1.5,2.5);

\draw[->, ultra thick] (3.49,2.5) to (3.5,2.5);
\draw[->, ultra thick] (4.49,2.5) to (4.5,2.5);

\draw[->, thick] (2.5,0) to (2.5,5);
\node at (2.5,5.4) {$\mathfrak{Im}(t)$};
\draw[->, ultra thick, color =myred] (2.5,3.76) to (2.5,3.75);
\draw[->, ultra thick, color = myred] (2.5,1.26) to (2.5,1.25);

\end{tikzpicture}
\caption{Visualisierung der Wick-Rotation}
\end{figure} 	
Die Wick-Rotation führt uns zu den sog. Euklidischen Pfadintegralen, deren Anwendungsgebiete gleich noch einmal etwas genauer vorgestellt werden.

\subsection{Statistische Physik}
In der Statistischen Physik sind wir vor allem an der Kenntnis der Zustandssummen $Z$ interessiert, da sich aus ihnen sämtliche interessanten Größen ableiten lassen. \\
Pfadintegrale liefern uns  eine neue Möglichkeit, diese Zustandssummen zu bestimmen. 
Wir betrachten nur Pfade, welche bei derselben Konfiguration starten und auch wieder enden:
\begin{align}
		Z = \int \mathcal{D}q \ \exp\left(-\frac{1}{\hbar}\mathcal{S}_{\text{eukl.}}[q]\right)
\end{align}
Dies entspricht genau der kanonischen Zustandssumme mit $\frac{1}{T} = \frac{k_B\tau}{\hbar}$ \\
Insgesamt gilt demnach:
\begin{align}
		Z = \int \dd q \ U(q,\beta\hbar;q) = \sum_k\bra{k} \int_q \dd q \ \operatorname{e}^{-\beta E_k}\ket{q}\braket{q}{k} = \sum_k \operatorname{e}^{-\beta E_k}
\end{align}
wobei wie üblich $\beta = \frac{1}{k_B T}$.
\subsection{Quantenfeldtheorie}
Auch in der Quantenfeldtheorie wird man häufig mit Pfadintegralen konfrontiert. Man berechnet zum Beispiel \textbf{Erwartungswerte} von Funktionalen $F[\phi]$:
\begin{align}
	\langle F \rangle = \frac{\int\mathcal{D}\phi \ F[\phi]\ \operatorname{e}^{-\frac{1}{\hbar}\mathcal{S}[\phi]}}{\int\mathcal{D}\phi' \ \operatorname{e}^{-\frac{1}{\hbar}\mathcal{S}[\phi']}}
\end{align}
Oder die \textbf{Korrelationsfunktionen}:
\begin{align}
\left\langle \phi(x_1) \phi(x_2) \ldots \phi(x_n)\right\rangle
=\frac{\int \mathcal D \phi \; \operatorname{e}^{-\frac{1}{\hbar}\mathcal{S}[\phi]}\phi(x_1)\ldots \phi(x_n)}{\int \mathcal D \phi' \; \operatorname{e}^{-\frac{1}{\hbar}\mathcal{S}[\phi']}}
\end{align}
In der Feldtheorie  wird hierbei die Variation aller möglichen Feldkonfigurationen anstelle der Pfade betrachtet. \\

Diese abschließend vorgestellten Beispiele sollen nur einen kleinen Ausblick auf die zahlreichen Anwendungen für Pfadintegrale in der theoretischen Physik liefern. 