\begin{center}

	\makeatletter
	\thispagestyle{plain}
	\LARGE\textbf{\@title} \\
	\vspace{2mm}
	\large\bfseries{\@author} \\
	\normalfont
	\vspace{2mm}
	\large{\@date} \\
	\vspace{2mm}
	\large{Institut für Theoretische Physik \\
		Universität Heidelberg} \\
	\makeatother
\end{center}

\normalsize

Dieser Vortrag entstand im Rahmen des Quantenmechanik-Seminars, organisiert von Prof. Wolschin am Institut für Theoretische Physik der Universität Heidelberg im Wintersemester 2018/2019. \\
Ziel des Vortrages ist es, eine alternative Formulierung der Quantenmechanik, den sog. Pfadintegral-Formulismus, zu motivieren und diesen aus dem bereits bekannten Konzept der Übergangsamplituden herzuleiten.  \\
Der Formalismus wird verwendet, um das Eigenenergie-Spektrum des Harmonischen Oszillators zu bestimmen und das Resultat mit den bereits bekannten Eigenenergien verglichen. \\
Ein Ausblick auf die zahlreichen Anwendungsgebiete der Pfadintegrale, vor allem in der Statistischen Physik und der Quantenfeldtheorie wird präsentiert. \\
Zu Beginn wird noch einmal das grundlegende Konzept des Wirkungsprinzips in der klassischen Mechanik sowie die bisher kennengelernten Interpretationen der Quantenmechanik nach Heisenberg und Schrödinger wiederholt.