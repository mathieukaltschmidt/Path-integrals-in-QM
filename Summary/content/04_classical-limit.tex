\section{Der klassische Grenzfall}
Um die Herleitung abzurunden, wollen wir nun den klassischen Grenzfall untersuchen, welcher sich im Limes $\hbar\rightarrow 0$ wiederfindet. \\
In diesem Fall oszilliert der Phasenfaktor $\phi[q] \sim \exp(\frac{\mathrm{i}}{\hbar} \mathcal{S}[q])$ sehr stark und es wird deutlich, dass nur Pfade in direkter Nähe zum Pfad stationärer Wirkung konstruktiv zur Übergangsamplitude beitragen, was genau dem klassischen Fall entspricht. Lösungen, welche weiter entfernt vom klassischen Pfad sind werden damit stark unterdrückt und liefern dementsprechend keinen Beitrag.  