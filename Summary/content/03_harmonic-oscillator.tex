\section{Der Harmonische Oszillator}
Wir wollen nun das Erlernte benutzen um das Eigenenergie-Spektrum für das  Potential des harmonischen Oszillators zu bestimmen. \\
Wir kennen den Lagrangian für dieses Problem: 
\begin{align}
	\mathcal{L} = \frac{m}{2}\dot{q}^2 - \frac{m\omega^2}{2}q^2
\end{align}
und parametrisieren alle Pfade als Abweichungen vom klassischen Pfad:
\begin{align}
	q(t) = q_{\text{cl}}(t) + \eta(t)
\end{align}
Die Wirkung des klassischen Beitrags lässt sich aus einer etwas länglichen aber nicht allzu schweren Rechnung bestimmen und wird an dieser Stelle als bekannt vorausgesetzt:
\begin{align}
	\mathcal{S}[q_{\text{cl}}] = \frac{m\omega}{2\sin(\omega T)}\left[(q_T^2+q_0^2)\cos(\omega T)- 2q_Tq_0\right]
\end{align}
